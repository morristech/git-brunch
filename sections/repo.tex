\section{Repo}
\stepcounter{subsection}
\begin{frame}
  \frametitle{About Repo}
  
  \begin{itemize}
    \item Google, 2008
    \item Android Open Source Project (AOSP)
    \item tool for managing multiple Git repositories
    \item integrates with Gerrit (Code Review)
  \end{itemize}
\end{frame}

\begin{frame}
  \frametitle{Running batch tasks}

  Usual Repo workflow
  \begin{itemize}
    \item create a feature branch over several repositories
    \item edit code across repositories to make the feature work
    \item commit the changes
    \item upload changes to Gerrit:
      \begin{itemize}
        \item creates one patch per commit
        \item in case there are several commits on the same project, they will be
          uploaded as dependent patches (on the same topic-branch)
      \end{itemize}
    \item code review happens ;-)
  \end{itemize}
\end{frame}

\begin{frame}{Repo commands (1)}
  \begin{description}
    \item[repo init -u url -m manifest] \hfill \\
      initialize a Repo workspace using a repository containing XML manifests
    \item[repo sync] \hfill \\
      update all repositories with upstream changes
    \item[repo start topic\_branch repo1] \hfill \\
      start a topic-branch on a given repository
    \item[repo prune topic\_branch] \hfill \\
      destroy a topic-branch
    \item[repo rebase] \hfill \\
      rebase a topic-branch on the mainline
    \item[repo upload] \hfill \\
      push local commits to Gerrit
  \end{description}
\end{frame}

\begin{frame}{Repo commands (2)}
  \begin{description}
    \item[repo status] \hfill \\
      status for local repositories
    \item[repo forall -c "command"] \hfill \\
      iterate on repositories, execute a (series of) command
    \item[repo forall -c "git clean -xdf; git reset -{}-hard HEAD"] \hfill \\
      efficient workspace cleanup before running \texttt{repo sync}
  \end{description}
\end{frame}
