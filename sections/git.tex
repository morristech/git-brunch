\section{Git}
\stepcounter{subsection}
\begin{frame}
  \frametitle{About Git}
  \begin{itemize}
    \item Linus Torvalds, 2005
    \item Linux kernel sources
    \item distributed
    \item focus on data integrity \& performance
    \item popularized by GitHub
  \end{itemize}
  \begin{block}{Other distributed SCMs}
    Mercurial (hg), GNU Bazaar (bzr), Fossil
  \end{block}
\end{frame}

\begin{frame}
  \frametitle{Git repositories (1)}

  Git is a distributed SCM, which means:
  \begin{itemize}
    \item every working copy \textit{(fork)} -either local or distant-
      is a repository by itself
    \item any repository holds a complete history (for a given branch)
    \item developments may happen across various networks/communities
    \item keeping track of upstream changes is a prerequisite
  \end{itemize}
\end{frame}

\begin{frame}
  \frametitle{Commits}

  \begin{itemize}
    \item code changes are stored as \textit{commits}
    \item a commit contains
      \begin{itemize}
        \item a description: what has been modified?
        \item context information: author, signature, etc.
        \item the actual code changes (diff)
      \end{itemize}
    \item and is identified by a unique \textit{revision} (SHA-1)
    \item commits are \textit{local} objects, and contain changes locally
      \textit{staged} until they are \textit{pushed} to a remote repository
      (if any)
      % ain't it subversive? :D
  \end{itemize}
\end{frame}

\begin{frame}
  \frametitle{Branches}

  Branching is the essence of distributed SCMs
  \begin{itemize}
    \item any branch is a working copy
    \item creation, merging and deletion are cheap operations
    \item analogy: pointers, chained lists
      \begin{itemize}
        \item a commit points to its parent commit
        \item a branch is a pointer to a given revision (commit)
        \item a repository's \textit{head} refers to the most recent
          commit of an existing branch
      \end{itemize}
  \end{itemize}
\end{frame}

\begin{frame}
  \frametitle{Branches are cool!}
  \begin{center}
    % OMG!!!
    \includegraphics[width=0.6\textwidth]{img/omgabranch.jpg}
    % A BRANCH!!!
  \end{center}
\end{frame}

\begin{frame}
  \frametitle{Git repositories (2): remotes}

  \begin{itemize}
    \item \textit{remote}: alias for a parent repository,
      either local or distant
    \item any number of remotes can be added
  \end{itemize}
  \begin{block}{Use cases}
    GitHub workflow, development outsourcing
  \end{block}
\end{frame}

\begin{frame}{Git commands (1): commit changes}
  \begin{description}
    \item[git add] \hfill \\
      stage changes to include them in the next commit
    \item[git clean] \hfill \\
      remove undesired (unversioned, ignored) changes
    \item[git rm] \hfill \\
      delete and unversion files
    \item[git commit] \hfill \\
      create a commit from staged changes
    \item[git commit -s] \hfill \\
      create a signed commit
    \item[git commit -{}-amend] \hfill \\
      alter the last commit: add/remove/modify files
  \end{description}
\end{frame}

\begin{frame}{Git commands (2): branches}
  \begin{description}
    \item[git branch] \hfill \\
      create \& delete branches
    \item[git tag] \hfill \\
      create \& delete tags
    \item[git pull] \hfill \\
      update branch information
    \item[git push remote branch] \hfill \\
      push local commits to a remote branch
    \item[git rebase remote/branch] \hfill \\
      rebase the current branch
    \item[git reset -{}-hard revision] \hfill \\
      reset the current branch to a reference revision
  \end{description}
\end{frame}

\begin{frame}{Git commands (3): checkout, the all-in-one}
  \begin{description}
    \item[git checkout revision] \hfill \\
      checkout a given revision: branch, tag, commit ID...
    \item[git checkout revision file1 file2...] \hfill \\
      checkout the selected files on a given revision
    \item[git checkout -{}- file1 file2...] \hfill \\
      unstage changes for the selected files
    \item[git checkout revision -b branch] \hfill \\
      create a branch from a given revision, and checkout it
  \end{description}
\end{frame}

\begin{frame}{Git commands (4): information}
  \begin{description}
    \item[git diff] \hfill \\
      show unstaged changes
    \item[git diff revision] \hfill \\
      show the differences with a given revision
    \item[git log] \hfill \\
      display the commit history
    \item[git show revision] \hfill \\
      show the commit message and the diff
    \item[git status] \hfill \\
      display local commits, unversioned changes...
  \end{description}
\end{frame}
