\section{Git}
\stepcounter{subsection}
\begin{frame}{About Git}
  \begin{itemize}
    \item Linus Torvalds, 2005
    \item Linux kernel sources
    \item distributed
    \item focus on data integrity \& performance
    \item popularized by GitHub
  \end{itemize}
  \begin{block}{Other distributed SCMs}
    Mercurial (hg), GNU Bazaar (bzr), Fossil
  \end{block}
\end{frame}

\begin{frame}{Git repositories (1)}
  Git is a distributed SCM, which means:
  \begin{itemize}
    \item every working copy \textit{(fork)} -either local or distant-
      is itself a fully functional repository
    \item any repository holds a complete history (for a given branch)
    \item developments may happen across various networks/communities
    \item easy to keep track of upstream changes
  \end{itemize}
\end{frame}

\begin{frame}{Commits}
  \begin{itemize}
    \item code changes are stored as \textit{commits}
    \item a commit contains
      \begin{itemize}
        \item a description: what has been modified?
        \item context information: author, signature, etc.
        \item the actual code changes (diff)
      \end{itemize}
    \item and is identified by a unique \textit{revision} (SHA-1)
    \item commits are \textit{local} objects, and contain changes locally
      \textit{staged} until they are \textit{pushed} to a remote repository
      (if any)
      % ain't it subversive?
  \end{itemize}
\end{frame}

\begin{frame}{Branches}
  Branching is the essence of distributed SCMs
  \begin{itemize}
    \item any branch is a working copy
    \item creation, merging and deletion are cheap operations
    \item analogy: pointers, chained lists
      \begin{itemize}
        \item a commit points to its parent commit
        \item a branch is a pointer to a given revision (commit)
        \item a repository's \textit{head} refers to the most recent
          commit of an existing branch
      \end{itemize}
  \end{itemize}
\end{frame}

\begin{frame}{Branching example 1}
  \begin{tikzpicture}
    % simple branching example
    \tikzset{commit/.style = {shape=circle,draw}}
    \tikzset{label/.style = {shape=rectangle,draw}}
    \tikzset{pre/.style = {->}}

    % master
    \node[commit] (m1)                 {m1};
    \node[commit] (m2) [right=of m1]   {m2}
       edge [pre] (m1);
    \node[commit] (m3) [right=of m2]   {m3}
       edge [pre] (m2);
    \node[commit] (m4) [right=of m3]   {m4}
       edge [pre] (m3);
    \node[label]  (master) [right=of m4] {master}
       edge (m4);

    % hotfix branch
    \node[commit] (h1) [above right=of m4]   {h1}
       edge [pre] (m4);
    \node[label]  (hotfix) [right=of h1] {hotfix}
       edge (h1);

    % feature branch
    \node[commit] (f1) [below right=of m3]   {f1}
       edge [pre] (m3);
    \node[commit] (f2) [right=of f1]   {f2}
       edge [pre] (f1);
    \node[label]  (feature) [right=of f2] {feature}
       edge (f2);
  \end{tikzpicture}
\end{frame}

\begin{frame}{Branching example 2}
  \begin{tikzpicture}
    % common use case: experimenting with stuff
    % - base api / interfaces
    % - 2 potential solutions
    \tikzset{commit/.style = {shape=circle,draw}}
    \tikzset{label/.style = {shape=rectangle,draw}}
    \tikzset{pre/.style = {->}}

    % master
    \node[commit] (m1)                 {m1};
    \node[commit] (m2) [right=of m1]   {m2}
       edge [pre] (m1);
    \node[label]  (master) [right=of m2] {master}
       edge (m2);

    % api / interfaces
    \node[commit] (a1) [below right=of m2]   {a1}
        edge [pre] (m2);
    \node[commit] (a2) [right=of a1]   {a2}
        edge [pre] (a1);
    \node[label]  (api) [right=of a2] {db/api}
        edge (a2);

    % solution 1
    \node[commit] (s1) [above right=of a2]   {s1}
        edge [pre] (a2);
    \node[label]  (sol1) [right=of s1] {db/sol1}
        edge (s1);

    % solution 2
    \node[commit] (s2) [below right=of a2]   {s2}
        edge [pre] (a2);
    \node[label]  (sol2) [right=of s2] {db/sol2}
        edge (s2);

  \end{tikzpicture}
\end{frame}

\iftoggle{GOODIES}{
  \begin{frame}{Branches are cool!}
    \begin{center}
      % OMG!!!
      \includegraphics[width=0.6\textwidth]{img/omgabranch.jpg}
      % A BRANCH!!!
    \end{center}
  \end{frame}
}{}

\begin{frame}{Git repositories (2): remotes}
  \begin{itemize}
    \item \textit{remote}: alias for a parent repository,
      either local or distant
    \item any number of remotes can be added
  \end{itemize}
  \begin{block}{Use cases}
    GitHub workflow, development outsourcing
  \end{block}
\end{frame}
